% Options for packages loaded elsewhere
\PassOptionsToPackage{unicode}{hyperref}
\PassOptionsToPackage{hyphens}{url}
%
\documentclass[
  x11names]{article}
\usepackage{amsmath,amssymb}
\usepackage{lmodern}
\usepackage{iftex}
\ifPDFTeX
  \usepackage[T1]{fontenc}
  \usepackage[utf8]{inputenc}
  \usepackage{textcomp} % provide euro and other symbols
\else % if luatex or xetex
  \usepackage{unicode-math}
  \defaultfontfeatures{Scale=MatchLowercase}
  \defaultfontfeatures[\rmfamily]{Ligatures=TeX,Scale=1}
\fi
% Use upquote if available, for straight quotes in verbatim environments
\IfFileExists{upquote.sty}{\usepackage{upquote}}{}
\IfFileExists{microtype.sty}{% use microtype if available
  \usepackage[]{microtype}
  \UseMicrotypeSet[protrusion]{basicmath} % disable protrusion for tt fonts
}{}
\makeatletter
\@ifundefined{KOMAClassName}{% if non-KOMA class
  \IfFileExists{parskip.sty}{%
    \usepackage{parskip}
  }{% else
    \setlength{\parindent}{0pt}
    \setlength{\parskip}{6pt plus 2pt minus 1pt}}
}{% if KOMA class
  \KOMAoptions{parskip=half}}
\makeatother
\usepackage{xcolor}
\usepackage[margin=1in]{geometry}
\usepackage{graphicx}
\makeatletter
\def\maxwidth{\ifdim\Gin@nat@width>\linewidth\linewidth\else\Gin@nat@width\fi}
\def\maxheight{\ifdim\Gin@nat@height>\textheight\textheight\else\Gin@nat@height\fi}
\makeatother
% Scale images if necessary, so that they will not overflow the page
% margins by default, and it is still possible to overwrite the defaults
% using explicit options in \includegraphics[width, height, ...]{}
\setkeys{Gin}{width=\maxwidth,height=\maxheight,keepaspectratio}
% Set default figure placement to htbp
\makeatletter
\def\fps@figure{htbp}
\makeatother
\setlength{\emergencystretch}{3em} % prevent overfull lines
\providecommand{\tightlist}{%
  \setlength{\itemsep}{0pt}\setlength{\parskip}{0pt}}
\setcounter{secnumdepth}{-\maxdimen} % remove section numbering
\usepackage{fontspec} \usepackage{titling} \pretitle{\begin{center} \vspace{-3cm}\includegraphics[width=\linewidth]{images/Base_info/logo.png}\LARGE\\} \posttitle{\end{center}} \usepackage{float} \usepackage{fancyhdr} \usepackage{ragged2e} \usepackage{caption} \usepackage{colortbl} \captionsetup[figure]{labelformat=empty} \arrayrulecolor{white} \pagestyle{fancy} \fancyhead[L,C]{} \fancypagestyle{plain}{\pagestyle{fancy}} \PassOptionsToPackage{dvipsnames,svgnames*,x11names*}{xcolor} \definecolor{ceil}{rgb}{0.57, 0.63, 0.81} \usepackage[export]{adjustbox} \usepackage{wrapfig} \usepackage{graphicx} \usepackage{caption}
\usepackage{booktabs}
\usepackage{longtable}
\usepackage{array}
\usepackage{multirow}
\usepackage{wrapfig}
\usepackage{float}
\usepackage{colortbl}
\usepackage{pdflscape}
\usepackage{tabu}
\usepackage{threeparttable}
\usepackage{threeparttablex}
\usepackage[normalem]{ulem}
\usepackage{makecell}
\usepackage{xcolor}
\ifLuaTeX
  \usepackage{selnolig}  % disable illegal ligatures
\fi
\IfFileExists{bookmark.sty}{\usepackage{bookmark}}{\usepackage{hyperref}}
\IfFileExists{xurl.sty}{\usepackage{xurl}}{} % add URL line breaks if available
\urlstyle{same} % disable monospaced font for URLs
\hypersetup{
  hidelinks,
  pdfcreator={LaTeX via pandoc}}

\author{}
\date{\vspace{-2.5em}Fecha de creación: 03 April, 2023}

\begin{document}

\setmainfont{Arial}
\setsansfont{Arial}
\setmonofont{Arial}

\newcommand\invisiblesection[1]{%
  \refstepcounter{section}%
  \addcontentsline{toc}{section}{\protect\numberline{\thesection}#1}%
  \sectionmark{#1}}

\fancyhead[R]{\textbf{http://doi.org/10.31687/SaremLR.19.196}}

%
  \refstepcounter{section}%
  \addcontentsline{toc}{section}{\protect\numberline{\thesection}GENERALIDADES}%
  \sectionmark{GENERALIDADES}
\vspace{-0.4cm}

\includegraphics[width=1\linewidth]{images/Base_info/logo}

\vspace{1cm}

\begin{minipage}{0.7\textwidth}
\vspace{0.3cm}
\fontsize{20}{24}\selectfont\textit{Cephalorhynchus eutropia}

\vspace{0.3cm}
\fontsize{30}{36}\selectfont Delfín chileno
\end{minipage}
\hspace{0.05\textwidth}
\begin{minipage}{0.25\textwidth}
\includegraphics[width=\textwidth]{images/vu.png}
\end{minipage}

\normalsize

\begin{figure}[H]

{\centering \includegraphics[width=0.35\linewidth]{photos/Blastocerus dichotomus} 

}

\caption{Fotos por Salvador Dali}\label{fig:image}
\end{figure}

\begin{center}\rule{0.5\linewidth}{0.5pt}\end{center}

\justifying

\textbf{Citar como:} Morgenthaler, Annick; Coscarella, Mariano A.;
García, Néstor A.. (2019). \emph{Cephalorhynchus eutropia}. En:
SAyDS--SAREM (eds.) Categorización 2019 de los mamíferos de Argentina
según su riesgo de extinción. Lista Roja de los mamíferos de Argentina.
\url{http://doi.org/10.31687/SaremLR.19.196}

\begin{center}\rule{0.5\linewidth}{0.5pt}\end{center}

\newpage

%
  \refstepcounter{section}%
  \addcontentsline{toc}{section}{\protect\numberline{\thesection}ÁREA DE DISTRIBUCIÓN ACTUAL}%
  \sectionmark{ÁREA DE DISTRIBUCIÓN ACTUAL}
\begin{table}[H]
\centering
\begin{tabular}[t]{>{\raggedright\arraybackslash}m{16cm}>{}m{16cm}}
\toprule
\cellcolor{ceil}{\textcolor{white}{\textbf{\rule{0pt}{14pt}ÁREA DE DISTRIBUCIÓN ACTUAL}}}\\
\bottomrule
\end{tabular}
\end{table}

\includegraphics[width=1\linewidth]{maps/Cetartiodactyla/Cephalorhynchus_eutropia}

%
  \refstepcounter{section}%
  \addcontentsline{toc}{section}{\protect\numberline{\thesection}CATEGORÍAS DE CONSERVACIÓN}%
  \sectionmark{CATEGORÍAS DE CONSERVACIÓN}
\begin{table}[H]
\centering
\begin{tabular}[t]{>{\raggedright\arraybackslash}m{16cm}>{}m{16cm}}
\toprule
\cellcolor{ceil}{\textcolor{white}{\textbf{\rule{0pt}{14pt}CATEGORÍAS DE CONSERVACIÓN}}}\\
\bottomrule
\end{tabular}
\end{table}

\vspace{-0.4cm}

\textbf{Categoría Nacional de Conservación 2019}

VU (Vulnerable)

\textbf{Criterios y subcriterios}

D2

\textbf{Justificación de la categorización}

Es una especie cuasi endémica de las costas y fiordos chilenos. En
Argentina se registra un único grupo ~en la ría de Puerto Deseado y
posiblemente la especie llegue ocasionalmente a aguas argentinas en el
canal de Beagle, pero no se tiene información que lo confirme. La
población de Puerto Deseado no cuenta con estimaciones de abundancia o
tendencia poblacional, sin embargo, desde el 2009 se registra la
presencia de un pequeño grupo reproductivo de aproximadamente 8--10
individuos. La principal amenaza para los individuos presentes en la Ría
Deseado, consiste en su fragmentación del resto de la población, que se
encuentra a más de 600 km de distancia, y su bajo número de individuos
maduros (entre 3 y 8). Sin un aporte de nuevos individuos la viabilidad
a largo plazo de este grupo sería poco probable.~Teniendo en cuenta el
tamaño poblacional y el bajo número de localidades, la especie califica
como Vulnerable (VU) según el criterio D2 porque existe una posibilidad
razonable de verse afectada por una amenaza futura que podría elevar al
taxón a En Peligro Crítico (CR) o Extinta (EX) en un tiempo muy corto.
Es necesario obtener nueva información de distribución y estado
poblacional.

\textbf{Categoría Res. SAyDS 1030/04}

VU (Vulnerable)

\textbf{Categorías nacionales de conservación previas (SAREM)}

\arrayrulecolor{white}

%
  \refstepcounter{section}%
  \addcontentsline{toc}{section}{\protect\numberline{\thesection}TAXONOMÍA Y NOMENCLATURA}%
  \sectionmark{TAXONOMÍA Y NOMENCLATURA}
\begin{table}[H]
\centering
\begin{tabular}[t]{>{\raggedright\arraybackslash}m{16cm}>{}m{16cm}}
\toprule
\cellcolor{ceil}{\textcolor{white}{\textbf{\rule{0pt}{14pt}TAXONOMÍA Y NOMENCLATURA}}}\\
\bottomrule
\end{tabular}
\end{table}

%
  \refstepcounter{section}%
  \addcontentsline{toc}{section}{\protect\numberline{\thesection}INFORMACIÓN RELEVANTE PARA LA EVALUACIÓN}%
  \sectionmark{INFORMACIÓN RELEVANTE PARA LA EVALUACIÓN}
\begin{table}[H]
\centering
\begin{tabular}[t]{>{\raggedright\arraybackslash}m{16cm}>{}m{16cm}}
\toprule
\cellcolor{ceil}{\textcolor{white}{\textbf{\rule{0pt}{14pt}INFORMACIÓN RELEVANTE PARA LA EVALUACIÓN}}}\\
\bottomrule
\end{tabular}
\end{table}

%
  \refstepcounter{section}%
  \addcontentsline{toc}{section}{\protect\numberline{\thesection}RANGO GEOGRÁFICO, OCURRENCIA Y ABUNDANCIA Y NOMENCLATURA}%
  \sectionmark{RANGO GEOGRÁFICO, OCURRENCIA Y ABUNDANCIA Y NOMENCLATURA}
\begin{table}[H]
\centering
\begin{tabular}[t]{>{\raggedright\arraybackslash}m{16cm}>{}m{16cm}}
\toprule
\cellcolor{ceil}{\textcolor{white}{\textbf{\rule{0pt}{14pt}RANGO GEOGRÁFICO, OCURRENCIA Y ABUNDANCIA Y NOMENCLATURA}}}\\
\bottomrule
\end{tabular}
\end{table}

%
  \refstepcounter{section}%
  \addcontentsline{toc}{section}{\protect\numberline{\thesection}DATOS MORFOMÉTRICOS}%
  \sectionmark{DATOS MORFOMÉTRICOS}
\begin{table}[H]
\centering
\begin{tabular}[t]{>{\raggedright\arraybackslash}m{16cm}>{}m{16cm}}
\toprule
\cellcolor{ceil}{\textcolor{white}{\textbf{\rule{0pt}{14pt}DATOS MORFOMÉTRICOS}}}\\
\bottomrule
\end{tabular}
\end{table}

%
  \refstepcounter{section}%
  \addcontentsline{toc}{section}{\protect\numberline{\thesection}RASGOS ETO-ECOLÓGICOS}%
  \sectionmark{RASGOS ETO-ECOLÓGICOS}
\begin{table}[H]
\centering
\begin{tabular}[t]{>{\raggedright\arraybackslash}m{16cm}>{}m{16cm}}
\toprule
\cellcolor{ceil}{\textcolor{white}{\textbf{\rule{0pt}{14pt}RASGOS ETO-ECOLÓGICOS}}}\\
\bottomrule
\end{tabular}
\end{table}

%
  \refstepcounter{section}%
  \addcontentsline{toc}{section}{\protect\numberline{\thesection}CONSERVACIÓN E INVESTIGACIÓN}%
  \sectionmark{CONSERVACIÓN E INVESTIGACIÓN}
\begin{table}[H]
\centering
\begin{tabular}[t]{>{\raggedright\arraybackslash}m{16cm}>{}m{16cm}}
\toprule
\cellcolor{ceil}{\textcolor{white}{\textbf{\rule{0pt}{14pt}CONSERVACIÓN E INVESTIGACIÓN}}}\\
\bottomrule
\end{tabular}
\end{table}

%
  \refstepcounter{section}%
  \addcontentsline{toc}{section}{\protect\numberline{\thesection}BIBLIOGRAFÍA}%
  \sectionmark{BIBLIOGRAFÍA}
\begin{table}[H]
\centering
\begin{tabular}[t]{>{\raggedright\arraybackslash}m{16cm}>{}m{16cm}}
\toprule
\cellcolor{ceil}{\textcolor{white}{\textbf{\rule{0pt}{14pt}BIBLIOGRAFÍA}}}\\
\bottomrule
\end{tabular}
\end{table}

\newpage

%
  \refstepcounter{section}%
  \addcontentsline{toc}{section}{\protect\numberline{\thesection}AUTORES}%
  \sectionmark{AUTORES}
\begin{table}[H]
\centering
\begin{tabular}[t]{>{\raggedright\arraybackslash}m{16cm}>{}m{16cm}}
\toprule
\cellcolor{ceil}{\textcolor{white}{\textbf{\rule{0pt}{14pt}AUTORES}}}\\
\bottomrule
\end{tabular}
\end{table}

\textbf{AUTORES}

\begin{tabu} to \linewidth {>{}l>{\raggedright\arraybackslash}p{2cm}>{\raggedright}X}
\toprule
\textbf{\cellcolor{gray!6}{Coscarella, Mariano A.}} & \cellcolor{gray!6}{} & \cellcolor{gray!6}{Laboratorio de Mamíferos Marinos, CESIMAR-CONICET, Puerto Madryn, Chubut, Argentina}\\
\textbf{García, Néstor A.} &  & Laboratorio de Mamíferos Marinos, Centro para el Estudio de Sistemas Marinos, Centro Nacional Patagónico (CESIMAR - CENPAT – CONICET), Chubut, Argentina\\
\textbf{\cellcolor{gray!6}{Morgenthaler, Annick}} & \cellcolor{gray!6}{} & \cellcolor{gray!6}{Centro de Investigaciones Puerto Deseado, Instituto de Ciencias Ambientales, Sustentabilidad y Recursos Naturales (ICASUR), Universidad Nacional de la Patagonia Austral, Puerto Deseado, Santa Cruz, Argentina}\\
\bottomrule
\end{tabu}

\end{document}
