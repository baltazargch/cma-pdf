% Options for packages loaded elsewhere
\PassOptionsToPackage{unicode}{hyperref}
\PassOptionsToPackage{hyphens}{url}
%
\documentclass[
  x11names]{article}
\usepackage{amsmath,amssymb}
\usepackage{lmodern}
\usepackage{iftex}
\ifPDFTeX
  \usepackage[T1]{fontenc}
  \usepackage[utf8]{inputenc}
  \usepackage{textcomp} % provide euro and other symbols
\else % if luatex or xetex
  \usepackage{unicode-math}
  \defaultfontfeatures{Scale=MatchLowercase}
  \defaultfontfeatures[\rmfamily]{Ligatures=TeX,Scale=1}
\fi
% Use upquote if available, for straight quotes in verbatim environments
\IfFileExists{upquote.sty}{\usepackage{upquote}}{}
\IfFileExists{microtype.sty}{% use microtype if available
  \usepackage[]{microtype}
  \UseMicrotypeSet[protrusion]{basicmath} % disable protrusion for tt fonts
}{}
\makeatletter
\@ifundefined{KOMAClassName}{% if non-KOMA class
  \IfFileExists{parskip.sty}{%
    \usepackage{parskip}
  }{% else
    \setlength{\parindent}{0pt}
    \setlength{\parskip}{6pt plus 2pt minus 1pt}}
}{% if KOMA class
  \KOMAoptions{parskip=half}}
\makeatother
\usepackage{xcolor}
\usepackage[margin=1in]{geometry}
\usepackage{graphicx}
\makeatletter
\def\maxwidth{\ifdim\Gin@nat@width>\linewidth\linewidth\else\Gin@nat@width\fi}
\def\maxheight{\ifdim\Gin@nat@height>\textheight\textheight\else\Gin@nat@height\fi}
\makeatother
% Scale images if necessary, so that they will not overflow the page
% margins by default, and it is still possible to overwrite the defaults
% using explicit options in \includegraphics[width, height, ...]{}
\setkeys{Gin}{width=\maxwidth,height=\maxheight,keepaspectratio}
% Set default figure placement to htbp
\makeatletter
\def\fps@figure{htbp}
\makeatother
\setlength{\emergencystretch}{3em} % prevent overfull lines
\providecommand{\tightlist}{%
  \setlength{\itemsep}{0pt}\setlength{\parskip}{0pt}}
\setcounter{secnumdepth}{-\maxdimen} % remove section numbering
\usepackage{fontspec} \usepackage{titling} \pretitle{\begin{center} \vspace{-3cm}\includegraphics[width=\linewidth]{images/Base_info/logo.png}\LARGE\\} \posttitle{\end{center}} \usepackage{float} \usepackage{fancyhdr} \usepackage{ragged2e} \usepackage{caption} \usepackage{colortbl} \captionsetup[figure]{labelformat=empty} \arrayrulecolor{white} \pagestyle{fancy} \fancyhead[L,C]{} \fancypagestyle{plain}{\pagestyle{fancy}} \PassOptionsToPackage{dvipsnames,svgnames*,x11names*}{xcolor} \definecolor{ceil}{rgb}{0.57, 0.63, 0.81} \usepackage[export]{adjustbox} \usepackage{wrapfig} \usepackage{graphicx} \usepackage{caption}
\usepackage{booktabs}
\usepackage{longtable}
\usepackage{array}
\usepackage{multirow}
\usepackage{wrapfig}
\usepackage{float}
\usepackage{colortbl}
\usepackage{pdflscape}
\usepackage{tabu}
\usepackage{threeparttable}
\usepackage{threeparttablex}
\usepackage[normalem]{ulem}
\usepackage{makecell}
\usepackage{xcolor}
\ifLuaTeX
  \usepackage{selnolig}  % disable illegal ligatures
\fi
\IfFileExists{bookmark.sty}{\usepackage{bookmark}}{\usepackage{hyperref}}
\IfFileExists{xurl.sty}{\usepackage{xurl}}{} % add URL line breaks if available
\urlstyle{same} % disable monospaced font for URLs
\hypersetup{
  hidelinks,
  pdfcreator={LaTeX via pandoc}}

\author{}
\date{\vspace{-2.5em}Fecha de creación: 03 April, 2023}

\begin{document}

\setmainfont{Arial}
\setsansfont{Arial}
\setmonofont{Arial}

\newcommand\invisiblesection[1]{%
  \refstepcounter{section}%
  \addcontentsline{toc}{section}{\protect\numberline{\thesection}#1}%
  \sectionmark{#1}}

\fancyhead[R]{\textbf{http://doi.org/10.31687/SaremLR.19.182}}

%
  \refstepcounter{section}%
  \addcontentsline{toc}{section}{\protect\numberline{\thesection}GENERALIDADES}%
  \sectionmark{GENERALIDADES}
\vspace{-0.4cm}

\includegraphics[width=1\linewidth]{images/Base_info/logo}

\vspace{1cm}

\begin{minipage}{0.7\textwidth}
\vspace{0.3cm}
\fontsize{20}{24}\selectfont\textit{Orcinus orca}

\vspace{0.3cm}
\fontsize{30}{36}\selectfont Orca
\end{minipage}
\hspace{0.05\textwidth}
\begin{minipage}{0.25\textwidth}
\includegraphics[width=\textwidth]{images/lc.png}
\end{minipage}

\normalsize

\begin{figure}[H]

{\centering \includegraphics[width=0.35\linewidth]{photos/Blastocerus dichotomus} 

}

\caption{Fotos por Salvador Dali}\label{fig:image}
\end{figure}

\begin{center}\rule{0.5\linewidth}{0.5pt}\end{center}

\justifying

\textbf{Citar como:} Coscarella, Mariano A.; Cáceres-Saez, Iris; Loizaga
de Castro, Rocío; García, Néstor A.. (2019). \emph{Orcinus orca}. En:
SAyDS--SAREM (eds.) Categorización 2019 de los mamíferos de Argentina
según su riesgo de extinción. Lista Roja de los mamíferos de Argentina.
\url{http://doi.org/10.31687/SaremLR.19.182}

\begin{center}\rule{0.5\linewidth}{0.5pt}\end{center}

\newpage

%
  \refstepcounter{section}%
  \addcontentsline{toc}{section}{\protect\numberline{\thesection}ÁREA DE DISTRIBUCIÓN ACTUAL}%
  \sectionmark{ÁREA DE DISTRIBUCIÓN ACTUAL}
\begin{table}[H]
\centering
\begin{tabular}[t]{>{\raggedright\arraybackslash}m{16cm}>{}m{16cm}}
\toprule
\cellcolor{ceil}{\textcolor{white}{\textbf{\rule{0pt}{14pt}ÁREA DE DISTRIBUCIÓN ACTUAL}}}\\
\bottomrule
\end{tabular}
\end{table}

\includegraphics[width=1\linewidth]{maps/Cetartiodactyla/Orcinus_orca}

%
  \refstepcounter{section}%
  \addcontentsline{toc}{section}{\protect\numberline{\thesection}CATEGORÍAS DE CONSERVACIÓN}%
  \sectionmark{CATEGORÍAS DE CONSERVACIÓN}
\begin{table}[H]
\centering
\begin{tabular}[t]{>{\raggedright\arraybackslash}m{16cm}>{}m{16cm}}
\toprule
\cellcolor{ceil}{\textcolor{white}{\textbf{\rule{0pt}{14pt}CATEGORÍAS DE CONSERVACIÓN}}}\\
\bottomrule
\end{tabular}
\end{table}

\vspace{-0.4cm}

\textbf{Categoría Nacional de Conservación 2019}

LC (Preocupación Menor)

\textbf{Criterios y subcriterios}

NA

\textbf{Justificación de la categorización}

Es una especie que puede ser observada en todo el mar argentino y es
abundante en la región antártica, aunque no existen estimaciones de
abundancia para gran parte de la costa argentina. Sin embargo, existe
evidencia indirecta que sugiere que los avistajes de orcas en el mar
argentino se incrementaron en los últimos 20 años (Crespo \& García
2016). En la actualidad no existen amenazas detectadas directas para la
especie. Por lo tanto, se concluye clasificarla como Preocupación Menor
(LC).

\textbf{Categoría Res. SAyDS 1030/04}

IC (Insuficientemente Conocida)

\textbf{Categorías nacionales de conservación previas (SAREM)}

\arrayrulecolor{white}

%
  \refstepcounter{section}%
  \addcontentsline{toc}{section}{\protect\numberline{\thesection}TAXONOMÍA Y NOMENCLATURA}%
  \sectionmark{TAXONOMÍA Y NOMENCLATURA}
\begin{table}[H]
\centering
\begin{tabular}[t]{>{\raggedright\arraybackslash}m{16cm}>{}m{16cm}}
\toprule
\cellcolor{ceil}{\textcolor{white}{\textbf{\rule{0pt}{14pt}TAXONOMÍA Y NOMENCLATURA}}}\\
\bottomrule
\end{tabular}
\end{table}

%
  \refstepcounter{section}%
  \addcontentsline{toc}{section}{\protect\numberline{\thesection}INFORMACIÓN RELEVANTE PARA LA EVALUACIÓN}%
  \sectionmark{INFORMACIÓN RELEVANTE PARA LA EVALUACIÓN}
\begin{table}[H]
\centering
\begin{tabular}[t]{>{\raggedright\arraybackslash}m{16cm}>{}m{16cm}}
\toprule
\cellcolor{ceil}{\textcolor{white}{\textbf{\rule{0pt}{14pt}INFORMACIÓN RELEVANTE PARA LA EVALUACIÓN}}}\\
\bottomrule
\end{tabular}
\end{table}

%
  \refstepcounter{section}%
  \addcontentsline{toc}{section}{\protect\numberline{\thesection}RANGO GEOGRÁFICO, OCURRENCIA Y ABUNDANCIA Y NOMENCLATURA}%
  \sectionmark{RANGO GEOGRÁFICO, OCURRENCIA Y ABUNDANCIA Y NOMENCLATURA}
\begin{table}[H]
\centering
\begin{tabular}[t]{>{\raggedright\arraybackslash}m{16cm}>{}m{16cm}}
\toprule
\cellcolor{ceil}{\textcolor{white}{\textbf{\rule{0pt}{14pt}RANGO GEOGRÁFICO, OCURRENCIA Y ABUNDANCIA Y NOMENCLATURA}}}\\
\bottomrule
\end{tabular}
\end{table}

%
  \refstepcounter{section}%
  \addcontentsline{toc}{section}{\protect\numberline{\thesection}DATOS MORFOMÉTRICOS}%
  \sectionmark{DATOS MORFOMÉTRICOS}
\begin{table}[H]
\centering
\begin{tabular}[t]{>{\raggedright\arraybackslash}m{16cm}>{}m{16cm}}
\toprule
\cellcolor{ceil}{\textcolor{white}{\textbf{\rule{0pt}{14pt}DATOS MORFOMÉTRICOS}}}\\
\bottomrule
\end{tabular}
\end{table}

%
  \refstepcounter{section}%
  \addcontentsline{toc}{section}{\protect\numberline{\thesection}RASGOS ETO-ECOLÓGICOS}%
  \sectionmark{RASGOS ETO-ECOLÓGICOS}
\begin{table}[H]
\centering
\begin{tabular}[t]{>{\raggedright\arraybackslash}m{16cm}>{}m{16cm}}
\toprule
\cellcolor{ceil}{\textcolor{white}{\textbf{\rule{0pt}{14pt}RASGOS ETO-ECOLÓGICOS}}}\\
\bottomrule
\end{tabular}
\end{table}

%
  \refstepcounter{section}%
  \addcontentsline{toc}{section}{\protect\numberline{\thesection}CONSERVACIÓN E INVESTIGACIÓN}%
  \sectionmark{CONSERVACIÓN E INVESTIGACIÓN}
\begin{table}[H]
\centering
\begin{tabular}[t]{>{\raggedright\arraybackslash}m{16cm}>{}m{16cm}}
\toprule
\cellcolor{ceil}{\textcolor{white}{\textbf{\rule{0pt}{14pt}CONSERVACIÓN E INVESTIGACIÓN}}}\\
\bottomrule
\end{tabular}
\end{table}

%
  \refstepcounter{section}%
  \addcontentsline{toc}{section}{\protect\numberline{\thesection}BIBLIOGRAFÍA}%
  \sectionmark{BIBLIOGRAFÍA}
\begin{table}[H]
\centering
\begin{tabular}[t]{>{\raggedright\arraybackslash}m{16cm}>{}m{16cm}}
\toprule
\cellcolor{ceil}{\textcolor{white}{\textbf{\rule{0pt}{14pt}BIBLIOGRAFÍA}}}\\
\bottomrule
\end{tabular}
\end{table}

\newpage

%
  \refstepcounter{section}%
  \addcontentsline{toc}{section}{\protect\numberline{\thesection}AUTORES}%
  \sectionmark{AUTORES}
\begin{table}[H]
\centering
\begin{tabular}[t]{>{\raggedright\arraybackslash}m{16cm}>{}m{16cm}}
\toprule
\cellcolor{ceil}{\textcolor{white}{\textbf{\rule{0pt}{14pt}AUTORES}}}\\
\bottomrule
\end{tabular}
\end{table}

\textbf{AUTORES}

\begin{tabu} to \linewidth {>{}l>{\raggedright\arraybackslash}p{2cm}>{\raggedright}X}
\toprule
\textbf{\cellcolor{gray!6}{Cáceres-Saez, Iris}} & \cellcolor{gray!6}{} & \cellcolor{gray!6}{Laboratorio de Ecología, Comportamiento y Mamíferos Marinos, División Mastozoología, Museo Argentino de Ciencias Naturales Bernardino Rivadavia (MACN-CONICET), CABA, Argentina}\\
\textbf{Coscarella, Mariano A.} &  & Laboratorio de Mamíferos Marinos, CESIMAR-CONICET, Puerto Madryn, Chubut, Argentina\\
\textbf{\cellcolor{gray!6}{García, Néstor A.}} & \cellcolor{gray!6}{} & \cellcolor{gray!6}{Laboratorio de Mamíferos Marinos, Centro para el Estudio de Sistemas Marinos, Centro Nacional Patagónico (CESIMAR - CENPAT – CONICET), Chubut, Argentina}\\
\textbf{Loizaga de Castro, Rocío} &  & Laboratorio de Mamíferos Marinos, Centro para el Estudio de Sistemas Marinos, Centro Nacional Patagónico (CESIMAR - CENPAT – CONICET)., Chubut, Argentina\\
\bottomrule
\end{tabu}

\end{document}
