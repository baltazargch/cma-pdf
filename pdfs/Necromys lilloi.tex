% Options for packages loaded elsewhere
\PassOptionsToPackage{unicode}{hyperref}
\PassOptionsToPackage{hyphens}{url}
%
\documentclass[
  x11names]{article}
\usepackage{amsmath,amssymb}
\usepackage{lmodern}
\usepackage{iftex}
\ifPDFTeX
  \usepackage[T1]{fontenc}
  \usepackage[utf8]{inputenc}
  \usepackage{textcomp} % provide euro and other symbols
\else % if luatex or xetex
  \usepackage{unicode-math}
  \defaultfontfeatures{Scale=MatchLowercase}
  \defaultfontfeatures[\rmfamily]{Ligatures=TeX,Scale=1}
\fi
% Use upquote if available, for straight quotes in verbatim environments
\IfFileExists{upquote.sty}{\usepackage{upquote}}{}
\IfFileExists{microtype.sty}{% use microtype if available
  \usepackage[]{microtype}
  \UseMicrotypeSet[protrusion]{basicmath} % disable protrusion for tt fonts
}{}
\makeatletter
\@ifundefined{KOMAClassName}{% if non-KOMA class
  \IfFileExists{parskip.sty}{%
    \usepackage{parskip}
  }{% else
    \setlength{\parindent}{0pt}
    \setlength{\parskip}{6pt plus 2pt minus 1pt}}
}{% if KOMA class
  \KOMAoptions{parskip=half}}
\makeatother
\usepackage{xcolor}
\usepackage[margin=1in]{geometry}
\usepackage{graphicx}
\makeatletter
\def\maxwidth{\ifdim\Gin@nat@width>\linewidth\linewidth\else\Gin@nat@width\fi}
\def\maxheight{\ifdim\Gin@nat@height>\textheight\textheight\else\Gin@nat@height\fi}
\makeatother
% Scale images if necessary, so that they will not overflow the page
% margins by default, and it is still possible to overwrite the defaults
% using explicit options in \includegraphics[width, height, ...]{}
\setkeys{Gin}{width=\maxwidth,height=\maxheight,keepaspectratio}
% Set default figure placement to htbp
\makeatletter
\def\fps@figure{htbp}
\makeatother
\setlength{\emergencystretch}{3em} % prevent overfull lines
\providecommand{\tightlist}{%
  \setlength{\itemsep}{0pt}\setlength{\parskip}{0pt}}
\setcounter{secnumdepth}{-\maxdimen} % remove section numbering
\usepackage{fontspec} \usepackage{titling} \pretitle{\begin{center} \vspace{-3cm}\includegraphics[width=\linewidth]{images/Base_info/logo.png}\LARGE\\} \posttitle{\end{center}} \usepackage{float} \usepackage{fancyhdr} \usepackage{ragged2e} \usepackage{caption} \usepackage{colortbl} \captionsetup[figure]{labelformat=empty} \arrayrulecolor{white} \pagestyle{fancy} \fancyhead[L,C]{} \fancypagestyle{plain}{\pagestyle{fancy}} \PassOptionsToPackage{dvipsnames,svgnames*,x11names*}{xcolor} \definecolor{ceil}{rgb}{0.57, 0.63, 0.81}
\usepackage{booktabs}
\usepackage{longtable}
\usepackage{array}
\usepackage{multirow}
\usepackage{wrapfig}
\usepackage{float}
\usepackage{colortbl}
\usepackage{pdflscape}
\usepackage{tabu}
\usepackage{threeparttable}
\usepackage{threeparttablex}
\usepackage[normalem]{ulem}
\usepackage{makecell}
\usepackage{xcolor}
\ifLuaTeX
  \usepackage{selnolig}  % disable illegal ligatures
\fi
\IfFileExists{bookmark.sty}{\usepackage{bookmark}}{\usepackage{hyperref}}
\IfFileExists{xurl.sty}{\usepackage{xurl}}{} % add URL line breaks if available
\urlstyle{same} % disable monospaced font for URLs
\hypersetup{
  pdftitle={Necromys lilloi},
  hidelinks,
  pdfcreator={LaTeX via pandoc}}

\title{\emph{Necromys lilloi}}
\usepackage{etoolbox}
\makeatletter
\providecommand{\subtitle}[1]{% add subtitle to \maketitle
  \apptocmd{\@title}{\par {\large #1 \par}}{}{}
}
\makeatother
\subtitle{\textbf{Ratón cavador de Lillo}}
\author{}
\date{\vspace{-2.5em}Fecha de creación: 03 April, 2023}

\begin{document}
\maketitle

\setmainfont{Calibri}
\setsansfont{Calibri}
\setmonofont{Calibri}

\fancyhead[R]{\textbf{http://doi.org/10.31687/SaremLR.19.263}}

\center

\textbf{Autores: Libardi, Gustavo Simões; Gómez Villafañe, Isabel}

\begin{figure}[H]

{\centering \includegraphics[width=0.35\linewidth]{photos/Necromys lilloi} 

}

\caption{Fotos por Salvador Dali}\label{fig:image}
\end{figure}

\begin{center}\includegraphics[width=0.7\linewidth]{images/scale-dd} \end{center}

\begin{center}\rule{0.5\linewidth}{0.5pt}\end{center}

\justifying

\textbf{Citar como:} Libardi, Gustavo Simões; Gómez Villafañe, Isabel.
(2019). \emph{Necromys lilloi}. En: SAyDS--SAREM (eds.) Categorización
2019 de los mamíferos de Argentina según su riesgo de extinción. Lista
Roja de los mamíferos de Argentina. Versión digital:
\url{http://cma.sarem.org.ar}.

\begin{center}\rule{0.5\linewidth}{0.5pt}\end{center}

\newpage

\begin{table}[H]
\centering
\begin{tabular}[t]{>{\raggedright\arraybackslash}m{16cm}>{}m{16cm}}
\toprule
\cellcolor{ceil}{\textcolor{white}{\textbf{\rule{0pt}{14pt}CATEGORÍAS DE CONSERVACIÓN}}}\\
\bottomrule
\end{tabular}
\end{table}

\vspace{-0.4cm}

\textbf{Categoría Nacional de Conservación 2019}

DD (Datos Insuficientes)

\textbf{Criterios y subcriterios}

NA

\textbf{Justificación de la categorización}

Esta especie endémica es conocida apenas para su localidad típica (Jayat
et al.~2016), donde fue capturada en pastizales del Chaco Seco. No se
conoce el rango de distribución geográfica en esta ecorregión, ni el
tamaño poblacional. Según la historia de captura en la localidad tipo,
la especie podría presentar fluctuaciones poblacionales. No existe
información acerca de su historia natural, del estado de sus poblaciónes
o de las amenazas que pueden afectarla. Por esa razón, la especie fue
categorizada como Datos Insuficientes~(DD). La especie aún no ha
sido~categorizada por IUCN.

\textbf{Categoría Res. SAyDS 1030/04}

NE (No Evaluada)

\textbf{Categorías nacionales de conservación previas (SAREM)}

\arrayrulecolor{white}

\begin{table}[H]
\centering
\begin{tabular}[t]{>{\raggedright\arraybackslash}m{16cm}>{}m{16cm}}
\toprule
\cellcolor{ceil}{\textcolor{white}{\textbf{\rule{0pt}{14pt}TAXONOMÍA Y NOMENCLATURA}}}\\
\bottomrule
\end{tabular}
\end{table}

\begin{table}[H]
\centering
\begin{tabular}[t]{>{\raggedright\arraybackslash}m{16cm}>{}m{16cm}}
\toprule
\cellcolor{ceil}{\textcolor{white}{\textbf{\rule{0pt}{14pt}INFORMACIÓN RELEVANTE PARA LA EVALUACIÓN}}}\\
\bottomrule
\end{tabular}
\end{table}

\begin{table}[H]
\centering
\begin{tabular}[t]{>{\raggedright\arraybackslash}m{16cm}>{}m{16cm}}
\toprule
\cellcolor{ceil}{\textcolor{white}{\textbf{\rule{0pt}{14pt}RANGO GEOGRÁFICO, OCURRENCIA Y ABUNDANCIA Y NOMENCLATURA}}}\\
\bottomrule
\end{tabular}
\end{table}

\begin{table}[H]
\centering
\begin{tabular}[t]{>{\raggedright\arraybackslash}m{16cm}>{}m{16cm}}
\toprule
\cellcolor{ceil}{\textcolor{white}{\textbf{\rule{0pt}{14pt}DATOS MORFOMÉTRICOS}}}\\
\bottomrule
\end{tabular}
\end{table}

\begin{table}[H]
\centering
\begin{tabular}[t]{>{\raggedright\arraybackslash}m{16cm}>{}m{16cm}}
\toprule
\cellcolor{ceil}{\textcolor{white}{\textbf{\rule{0pt}{14pt}RASGOS ETO-ECOLÓGICOS}}}\\
\bottomrule
\end{tabular}
\end{table}

\begin{table}[H]
\centering
\begin{tabular}[t]{>{\raggedright\arraybackslash}m{16cm}>{}m{16cm}}
\toprule
\cellcolor{ceil}{\textcolor{white}{\textbf{\rule{0pt}{14pt}CONSERVACIÓN E INVESTIGACIÓN}}}\\
\bottomrule
\end{tabular}
\end{table}

\begin{table}[H]
\centering
\begin{tabular}[t]{>{\raggedright\arraybackslash}m{16cm}>{}m{16cm}}
\toprule
\cellcolor{ceil}{\textcolor{white}{\textbf{\rule{0pt}{14pt}BIBLIOGRAFÍA}}}\\
\bottomrule
\end{tabular}
\end{table}

\begin{table}[H]
\centering
\begin{tabular}[t]{>{\raggedright\arraybackslash}m{16cm}>{}m{16cm}}
\toprule
\cellcolor{ceil}{\textcolor{white}{\textbf{\rule{0pt}{14pt}AUTORES}}}\\
\bottomrule
\end{tabular}
\end{table}

\end{document}
