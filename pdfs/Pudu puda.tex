% Options for packages loaded elsewhere
\PassOptionsToPackage{unicode}{hyperref}
\PassOptionsToPackage{hyphens}{url}
%
\documentclass[
  x11names]{article}
\usepackage{amsmath,amssymb}
\usepackage{lmodern}
\usepackage{iftex}
\ifPDFTeX
  \usepackage[T1]{fontenc}
  \usepackage[utf8]{inputenc}
  \usepackage{textcomp} % provide euro and other symbols
\else % if luatex or xetex
  \usepackage{unicode-math}
  \defaultfontfeatures{Scale=MatchLowercase}
  \defaultfontfeatures[\rmfamily]{Ligatures=TeX,Scale=1}
\fi
% Use upquote if available, for straight quotes in verbatim environments
\IfFileExists{upquote.sty}{\usepackage{upquote}}{}
\IfFileExists{microtype.sty}{% use microtype if available
  \usepackage[]{microtype}
  \UseMicrotypeSet[protrusion]{basicmath} % disable protrusion for tt fonts
}{}
\makeatletter
\@ifundefined{KOMAClassName}{% if non-KOMA class
  \IfFileExists{parskip.sty}{%
    \usepackage{parskip}
  }{% else
    \setlength{\parindent}{0pt}
    \setlength{\parskip}{6pt plus 2pt minus 1pt}}
}{% if KOMA class
  \KOMAoptions{parskip=half}}
\makeatother
\usepackage{xcolor}
\usepackage[margin=1in]{geometry}
\usepackage{graphicx}
\makeatletter
\def\maxwidth{\ifdim\Gin@nat@width>\linewidth\linewidth\else\Gin@nat@width\fi}
\def\maxheight{\ifdim\Gin@nat@height>\textheight\textheight\else\Gin@nat@height\fi}
\makeatother
% Scale images if necessary, so that they will not overflow the page
% margins by default, and it is still possible to overwrite the defaults
% using explicit options in \includegraphics[width, height, ...]{}
\setkeys{Gin}{width=\maxwidth,height=\maxheight,keepaspectratio}
% Set default figure placement to htbp
\makeatletter
\def\fps@figure{htbp}
\makeatother
\setlength{\emergencystretch}{3em} % prevent overfull lines
\providecommand{\tightlist}{%
  \setlength{\itemsep}{0pt}\setlength{\parskip}{0pt}}
\setcounter{secnumdepth}{-\maxdimen} % remove section numbering
\usepackage{fontspec} \usepackage{titling} \pretitle{\begin{center} \vspace{-3cm}\includegraphics[width=\linewidth]{images/Base_info/logo.png}\LARGE\\} \posttitle{\end{center}} \usepackage{float} \usepackage{fancyhdr} \usepackage{ragged2e} \usepackage{caption} \usepackage{colortbl} \captionsetup[figure]{labelformat=empty} \arrayrulecolor{white} \pagestyle{fancy} \fancyhead[L,C]{} \fancypagestyle{plain}{\pagestyle{fancy}} \PassOptionsToPackage{dvipsnames,svgnames*,x11names*}{xcolor} \definecolor{ceil}{rgb}{0.57, 0.63, 0.81} \usepackage[export]{adjustbox} \usepackage{wrapfig} \usepackage{graphicx} \usepackage{caption}
\usepackage{booktabs}
\usepackage{longtable}
\usepackage{array}
\usepackage{multirow}
\usepackage{wrapfig}
\usepackage{float}
\usepackage{colortbl}
\usepackage{pdflscape}
\usepackage{tabu}
\usepackage{threeparttable}
\usepackage{threeparttablex}
\usepackage[normalem]{ulem}
\usepackage{makecell}
\usepackage{xcolor}
\ifLuaTeX
  \usepackage{selnolig}  % disable illegal ligatures
\fi
\IfFileExists{bookmark.sty}{\usepackage{bookmark}}{\usepackage{hyperref}}
\IfFileExists{xurl.sty}{\usepackage{xurl}}{} % add URL line breaks if available
\urlstyle{same} % disable monospaced font for URLs
\hypersetup{
  hidelinks,
  pdfcreator={LaTeX via pandoc}}

\author{}
\date{\vspace{-2.5em}Fecha de creación: 03 April, 2023}

\begin{document}

\setmainfont{Arial}
\setsansfont{Arial}
\setmonofont{Arial}

\newcommand\invisiblesection[1]{%
  \refstepcounter{section}%
  \addcontentsline{toc}{section}{\protect\numberline{\thesection}#1}%
  \sectionmark{#1}}

\fancyhead[R]{\textbf{http://doi.org/10.31687/SaremLR.19.214}}

%
  \refstepcounter{section}%
  \addcontentsline{toc}{section}{\protect\numberline{\thesection}GENERALIDADES}%
  \sectionmark{GENERALIDADES}
\vspace{-0.4cm}

\includegraphics[width=1\linewidth]{images/Base_info/logo}

\vspace{1cm}

\begin{minipage}{0.7\textwidth}
\vspace{0.3cm}
\fontsize{20}{24}\selectfont\textit{Pudu puda}

\vspace{0.3cm}
\fontsize{30}{36}\selectfont Pudú
\end{minipage}
\hspace{0.05\textwidth}
\begin{minipage}{0.25\textwidth}
\includegraphics[width=\textwidth]{images/vu.png}
\end{minipage}

\normalsize

\begin{figure}[H]

{\centering \includegraphics[width=0.35\linewidth]{photos/Blastocerus dichotomus} 

}

\caption{Fotos por Salvador Dali}\label{fig:image}
\end{figure}

\begin{center}\rule{0.5\linewidth}{0.5pt}\end{center}

\justifying

\textbf{Citar como:} Ballari, Sebastián A.; Pastore, Hernán; Varela,
Diego. (2019). \emph{Pudu puda}. En: SAyDS--SAREM (eds.) Categorización
2019 de los mamíferos de Argentina según su riesgo de extinción. Lista
Roja de los mamíferos de Argentina.
\url{http://doi.org/10.31687/SaremLR.19.214}

\begin{center}\rule{0.5\linewidth}{0.5pt}\end{center}

\newpage

%
  \refstepcounter{section}%
  \addcontentsline{toc}{section}{\protect\numberline{\thesection}ÁREA DE DISTRIBUCIÓN ACTUAL}%
  \sectionmark{ÁREA DE DISTRIBUCIÓN ACTUAL}
\begin{table}[H]
\centering
\begin{tabular}[t]{>{\raggedright\arraybackslash}m{16cm}>{}m{16cm}}
\toprule
\cellcolor{ceil}{\textcolor{white}{\textbf{\rule{0pt}{14pt}ÁREA DE DISTRIBUCIÓN ACTUAL}}}\\
\bottomrule
\end{tabular}
\end{table}

\includegraphics[width=1\linewidth]{maps/Cetartiodactyla/Pudu_puda}

%
  \refstepcounter{section}%
  \addcontentsline{toc}{section}{\protect\numberline{\thesection}CATEGORÍAS DE CONSERVACIÓN}%
  \sectionmark{CATEGORÍAS DE CONSERVACIÓN}
\begin{table}[H]
\centering
\begin{tabular}[t]{>{\raggedright\arraybackslash}m{16cm}>{}m{16cm}}
\toprule
\cellcolor{ceil}{\textcolor{white}{\textbf{\rule{0pt}{14pt}CATEGORÍAS DE CONSERVACIÓN}}}\\
\bottomrule
\end{tabular}
\end{table}

\vspace{-0.4cm}

\textbf{Categoría Nacional de Conservación 2019}

VU (Vulnerable)

\textbf{Criterios y subcriterios}

A3cde+B1ab(ii,iii)

\textbf{Justificación de la categorización}

El pudú es una especie endémica de los bosques templados y costeros
sudamericanos de Argentina y Chile. En Argentina habita únicamente
bosques húmedos, templados y fríos con estrato arbustivo denso. Su rango
de distribución es acotado y se extiende desde el sudoeste de la
provincia de Neuquén, norte del Parque Nacional Lanín, hasta el noroeste
de Chubut. A pesar de que ladistribución del pudú está bajo la
protección de áreas protegidas (Argentina \textasciitilde66\%), se
sospecha una reducción futura en el tamaño poblacional mayor al 30\%
como consecuencia de numerosas amenazas potenciales y crecientes que
podrían afectar seriamente su estatus de conservación actual. Las
amenazas para esta especie en Argentina incluyen la pérdida y
fragmentación del hábitat, los atropellamientos en rutas, la caza ilegal
y el impacto de especies exóticas invasoras (EEI). Las EEI, que
representan importantes agentes de cambio a nivel global, son unas de
las amenazas más serias para este cérvido. En su distribución, esta
especie coexiste con ganado vacuno (Bos taurus), jabalí (Sus scrofa) y
ciervo colorado (Cervus elaphus), que representan competidores
potenciales por espacio y recursos para herbívoros nativos como el pudú.
Por otro lado, la presencia de perros (Canis lupus familiaris) sueltos,
que puede implicar el acoso y la depredación hacia el pudú, se indica
como una de las amenazas directas más importantes de este cérvido.
Además estas EEI son potenciales vectores de enfermedades que pueden
afectar el estado sanitario del pudú. Finalmente, la modificación del
hábitat provocada por intervención humana (rutas, deforestación, uso
turístico), sumado a la provocada por los impactos de las EEI,
representa una de las causas principales de la declinación del pudú por
pérdida y degradación de hábitat. Por lo tanto, se considera a la
especie en la categoría Vulnerable (VU), teniendo en cuenta el criterio
A3 de reducción del tamaño población a futuro y por el criterio B1,
debido a una extensión de presencia (EOO) menor a 20.000 km2 y menos de
10 localidades.

\textbf{Categoría Res. SAyDS 1030/04}

VU (Vulnerable)

\textbf{Categorías nacionales de conservación previas (SAREM)}

\arrayrulecolor{white}

%
  \refstepcounter{section}%
  \addcontentsline{toc}{section}{\protect\numberline{\thesection}TAXONOMÍA Y NOMENCLATURA}%
  \sectionmark{TAXONOMÍA Y NOMENCLATURA}
\begin{table}[H]
\centering
\begin{tabular}[t]{>{\raggedright\arraybackslash}m{16cm}>{}m{16cm}}
\toprule
\cellcolor{ceil}{\textcolor{white}{\textbf{\rule{0pt}{14pt}TAXONOMÍA Y NOMENCLATURA}}}\\
\bottomrule
\end{tabular}
\end{table}

%
  \refstepcounter{section}%
  \addcontentsline{toc}{section}{\protect\numberline{\thesection}INFORMACIÓN RELEVANTE PARA LA EVALUACIÓN}%
  \sectionmark{INFORMACIÓN RELEVANTE PARA LA EVALUACIÓN}
\begin{table}[H]
\centering
\begin{tabular}[t]{>{\raggedright\arraybackslash}m{16cm}>{}m{16cm}}
\toprule
\cellcolor{ceil}{\textcolor{white}{\textbf{\rule{0pt}{14pt}INFORMACIÓN RELEVANTE PARA LA EVALUACIÓN}}}\\
\bottomrule
\end{tabular}
\end{table}

%
  \refstepcounter{section}%
  \addcontentsline{toc}{section}{\protect\numberline{\thesection}RANGO GEOGRÁFICO, OCURRENCIA Y ABUNDANCIA Y NOMENCLATURA}%
  \sectionmark{RANGO GEOGRÁFICO, OCURRENCIA Y ABUNDANCIA Y NOMENCLATURA}
\begin{table}[H]
\centering
\begin{tabular}[t]{>{\raggedright\arraybackslash}m{16cm}>{}m{16cm}}
\toprule
\cellcolor{ceil}{\textcolor{white}{\textbf{\rule{0pt}{14pt}RANGO GEOGRÁFICO, OCURRENCIA Y ABUNDANCIA Y NOMENCLATURA}}}\\
\bottomrule
\end{tabular}
\end{table}

%
  \refstepcounter{section}%
  \addcontentsline{toc}{section}{\protect\numberline{\thesection}DATOS MORFOMÉTRICOS}%
  \sectionmark{DATOS MORFOMÉTRICOS}
\begin{table}[H]
\centering
\begin{tabular}[t]{>{\raggedright\arraybackslash}m{16cm}>{}m{16cm}}
\toprule
\cellcolor{ceil}{\textcolor{white}{\textbf{\rule{0pt}{14pt}DATOS MORFOMÉTRICOS}}}\\
\bottomrule
\end{tabular}
\end{table}

%
  \refstepcounter{section}%
  \addcontentsline{toc}{section}{\protect\numberline{\thesection}RASGOS ETO-ECOLÓGICOS}%
  \sectionmark{RASGOS ETO-ECOLÓGICOS}
\begin{table}[H]
\centering
\begin{tabular}[t]{>{\raggedright\arraybackslash}m{16cm}>{}m{16cm}}
\toprule
\cellcolor{ceil}{\textcolor{white}{\textbf{\rule{0pt}{14pt}RASGOS ETO-ECOLÓGICOS}}}\\
\bottomrule
\end{tabular}
\end{table}

%
  \refstepcounter{section}%
  \addcontentsline{toc}{section}{\protect\numberline{\thesection}CONSERVACIÓN E INVESTIGACIÓN}%
  \sectionmark{CONSERVACIÓN E INVESTIGACIÓN}
\begin{table}[H]
\centering
\begin{tabular}[t]{>{\raggedright\arraybackslash}m{16cm}>{}m{16cm}}
\toprule
\cellcolor{ceil}{\textcolor{white}{\textbf{\rule{0pt}{14pt}CONSERVACIÓN E INVESTIGACIÓN}}}\\
\bottomrule
\end{tabular}
\end{table}

%
  \refstepcounter{section}%
  \addcontentsline{toc}{section}{\protect\numberline{\thesection}BIBLIOGRAFÍA}%
  \sectionmark{BIBLIOGRAFÍA}
\begin{table}[H]
\centering
\begin{tabular}[t]{>{\raggedright\arraybackslash}m{16cm}>{}m{16cm}}
\toprule
\cellcolor{ceil}{\textcolor{white}{\textbf{\rule{0pt}{14pt}BIBLIOGRAFÍA}}}\\
\bottomrule
\end{tabular}
\end{table}

\newpage

%
  \refstepcounter{section}%
  \addcontentsline{toc}{section}{\protect\numberline{\thesection}AUTORES}%
  \sectionmark{AUTORES}
\begin{table}[H]
\centering
\begin{tabular}[t]{>{\raggedright\arraybackslash}m{16cm}>{}m{16cm}}
\toprule
\cellcolor{ceil}{\textcolor{white}{\textbf{\rule{0pt}{14pt}AUTORES}}}\\
\bottomrule
\end{tabular}
\end{table}

\textbf{AUTORES}

\begin{tabu} to \linewidth {>{}l>{\raggedright\arraybackslash}p{2cm}>{\raggedright}X}
\toprule
\textbf{\cellcolor{gray!6}{Ballari, Sebastián A.}} & \cellcolor{gray!6}{} & \cellcolor{gray!6}{CENAC-APN, Parque Nacional Nahuel Huapi-CONICET, Bariloche, Río Negro, Argentina}\\
\textbf{Pastore, Hernán} &  & Dirección Regional Patagonia Norte, Administración de Parques Nacionales, Bariloche, Río Negro, Argentina\\
\textbf{\cellcolor{gray!6}{Varela, Diego}} & \cellcolor{gray!6}{} & \cellcolor{gray!6}{Instituto de Biología Subtropical (IBS), CONICET-Universidad Nacional de Misiones y Centro de Investigaciones del Bosque Atlántico (CeIBA), Puerto Iguazú, Misiones, Argentina}\\
\bottomrule
\end{tabu}

\textbf{COLABORADORES}

\begin{tabu} to \linewidth {>{}l>{\raggedright\arraybackslash}p{2cm}>{\raggedright}X}
\toprule
\textbf{\cellcolor{gray!6}{Merino, Mariano L.}} & \cellcolor{gray!6}{} & \cellcolor{gray!6}{Centro de Bioinvestigaciones, Centro de Investigaciones y Transferencia del Noroeste de la Provincia de Buenos Aires (CIT-NOBA), UNNOBA-CONICET, Pergamino, Buenos Aires, Argentina}\\
\bottomrule
\end{tabu}

\end{document}
